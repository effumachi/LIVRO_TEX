\documentclass{article}
\usepackage[utf8]{inputenc}
\usepackage[a4paper, left=20mm, top=15mm, right=20mm, bottom=15mm]{geometry}
\usepackage{graphicx}
\usepackage{enumitem}
\usepackage{amsmath} 
\usepackage{amssymb}
\usepackage{tikz}
\usepackage{pgfplots}

\newcommand{\resposta}{\hfill\makebox[0pt][r]{\scriptsize\textit{Resposta:\rule{5cm}{.1pt}}}\\ \vspace{.1cm}}
\newcommand{\um}{(1,0 ponto) }
\newcommand{\dois}{(2,0 pontos) }
\newcommand{\tres}{(3,0 pontos) }
\begin{document}
\pagestyle{empty}

\begin{center}\textbf{\Large{Revisão Álgebra Linear}}\end{center}
\begin{center}\textbf{\Large{Prof. Dr. Fumachi}}\end{center}
\begin{center}\textbf{\Large{01/10/2019}}\end{center}

\textbf{{Questões}}

\vspace{0.25 cm}
\noindent 1. Determine $x$ e $y$ de modo que as matrizes abaixo sejam iguais:
\begin{equation*}\label{key}
\mathbb{A} = \left[ \begin{matrix} 1&2 \\ 4&x+5 \end{matrix} \right] \qquad \mathbb{B} = \left[ \begin{matrix} y-3&2 \\ 4&10 \end{matrix} \right]
\end{equation*}

%-------------------------------------
\noindent 2. Calcule $\mathbb{A} \cdot \mathbb{B}$ sendo $\mathbb{A} = \left[ \begin{matrix} 1&2 \\ 4&2 \end{matrix} \right]$ e $\mathbb{B} = \left[ \begin{matrix} 0&2 \\ 3&1 \end{matrix} \right]$

%--------------------------------------
%-------------------------------------
\noindent 3. Calcule $2 \mathbb{A} +3 \mathbb{B}$ sendo $\mathbb{A} = \left[ \begin{matrix} -1&0 \\ 1&2 \end{matrix} \right]$ e $\mathbb{B} = \left[ \begin{matrix} 0&2 \\ 1&-1 \end{matrix} \right]$

%--------------------------------------
%-------------------------------------
\noindent 4. Calcule $\mathbb{A} - \mathbb{B}$ sendo $\mathbb{A} = \left[ \begin{matrix} 4&5 \\ 0&2 \end{matrix} \right]$ e $\mathbb{B} = \left[ \begin{matrix} 0&2 \\ 3&1 \end{matrix} \right]$

%--------------------------------------
%-------------------------------------
\noindent 5. Calcule $\mathrm{det}(\mathbb{A})$ sendo $\mathbb{A} = \left[ \begin{matrix} 1&2 \\ 4&2 \end{matrix} \right]$.

%--------------------------------------
%-------------------------------------
\noindent 6. Calcule $\mathrm{det}(\mathbb{A})$ sendo $\mathbb{A} = \left[ \begin{matrix} 1&2&3 \\ 4&2&1 \\ -1&2&0 \end{matrix} \right]$.

%--------------------------------------
%-------------------------------------
\noindent 7. Calcule $\mathrm{det}(\mathbb{A})$ sendo $\mathbb{A} = \left[ \begin{matrix} 1&3&0&1 \\ 4&2&0&0 \\ -1&3&0&1 \\ 4&3&2&0 \end{matrix} \right]$.

%--------------------------------------
%-------------------------------------
\noindent 8. Calcule a área da figura abaixo:

\begin{tikzpicture}[scale=1.0]
		% Estilos
		\tikzstyle {every pin} = [ rectangle, rounded corners=1pt, font=\large ]
		\begin{axis}
		[
		grid, grid style=dashed,
		grid = both,
		ymin=-.5,ymax=6.2,
		xmin=-.5,xmax=6.2,
		%xlabel=$x$,
		%ylabel=$f(x)$,
		width=10cm,
		axis on top=true,
		axis x line=middle,
		axis y line=middle,
		%transpose legend,
		%legend columns = 1,
		%legend style = {at={0.5,-.1},anchor=north}
		legend pos= north east, %outer
		]
		%\draw[blue, very thick] (50,50) rectangle (550,550);
		\path[draw, fill=gray!20, very thick] (50,50) rectangle (550,550);
		%\draw[orange, ultra thick] (150,150) -- (450,150) -- (150,450) -- cycle;
		\path[draw, fill=white!10, ultra thick] (150,150) -- (450,150) -- (150,450) -- cycle;
		% Desenha a função
		%\addplot [blue,line width = 1, smooth, domain=-1:3] {2*x +1}; %\addlegendentry{x}
		
		% Pontos de interesse
		\node [fill=none, circle, scale=0.75, pin=45:{$A_h$}] at (axis cs: 3.5,3.5) {};
		%\node [fill=red, circle, scale=0.5, pin=180:{$(0,1)$}] at (axis cs: 0,1) {};
		%\node [fill=blue, circle, scale=0.1, pin=135:{$f(x)=2x+1$}] at (axis cs: 2.75,6.5) {};
		%\legend {$f(x)=2x+1$}
		\end{axis}
\end{tikzpicture}

%--------------------------------------


\par\noindent\rule{\textwidth}{1pt}

\end{document}