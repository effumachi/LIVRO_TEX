\section{Exercícios}\index{Exercícios}

\begin{exercise}
	Determine os valores de $x$ e $y$ de modo que as matrizes abaixo sejam iguais:
	
	\begin{ceqn}
		\begin{align*}
		\mathbb{A}_{2}=\begin{bmatrix}1 & x+10 \\ 3 & 5 \end{bmatrix} \quad \mathrm{e} \quad \mathbb{B}_{2}=\begin{bmatrix}1 & 25 \\ y-7 & 5 \end{bmatrix}
		\end{align*}
	\end{ceqn}
\end{exercise}

\begin{exercise}
	Considere as matrizes:
	\begin{ceqn}
		\begin{align*}
				\mathbb{A}=\begin{bmatrix}1 & 1 \\ 3 & 5 \end{bmatrix} \quad \mathrm{e} \quad \mathbb{B}=\begin{bmatrix}1 & 25 \\ -7 & 5 \end{bmatrix}
		\end{align*}
	\end{ceqn}
	Calcule:
	\begin{itemize}
		\item[a.]{$2 \mathbb{A}+3\mathbb{B}$}
		\item[b.]{$-\mathbb{A}+\frac{1}{2}\mathbb{B}$}
		\item[c.]{$\mathbb{A} \cdot \mathbb{B}$}
		\item[d.]{$\mathbb{B} \cdot \mathbb{A}$}
		\item[e.]{As matrizes $\mathbb{A} \cdot \mathbb{B}$ e $\mathbb{B} \cdot \mathbb{A}$ são iguais? Por quê?}
	\end{itemize}
\end{exercise}

\begin{exercise}
	Calcule o Determinante das matrizes abaixo:
	\begin{itemize}
		\item[a.]{
			\begin{align*}
			\mathbb{A}=\begin{bmatrix}1 & 1 \\ 3 & 5 \end{bmatrix}
			\end{align*}
			}
		\item[b.]{
			\begin{align*}
			\mathbb{A}=\begin{bmatrix}1 & 1 & 2 \\ 3 & 5 & -1 \\ -2 & 3 & 4 \end{bmatrix}
			\end{align*}
			}
		\item[c.]{
			\begin{align*}
			\mathbb{A}=\begin{bmatrix}1 & 1 & 0 & 2 \\ 3 & 5 & 0 & -10 \\ -11 & 1 & 1 & 7 \\ 6 & 2 & 0 & -10 \end{bmatrix}
			\end{align*}
			}
	\end{itemize}
\end{exercise}

\begin{exercise}
	Uma empresa produz e vende pneus automotivos. Sabendo que o custo envolvido na produção de 100 pneus seja igual a R\$ 2300,00 e o preço de venda de cada unidade igual a R\$ 730,00, determine o ponto de equilíbrio dessa produção supondo que o custo fixo da empresa seja igual a R\$ 3700,00.
\end{exercise}

\begin{exercise}
	\label{ex55}
	Calcule o sistema abaixo usando os métodos de Cramer e Escalonamento.
	\begin{itemize}
		\item[a.]{
			\begin{align*}
				\left \{
				\begin{matrix}
				x+y &=10 \\
				x-y &=0\\
				\end{matrix}
				\right .
			\end{align*}
			}
		\item[b.]{
			\begin{align*}
				\left \{
				\begin{matrix}
				x+y &=10 \\
				2x+2y &=0\\
				\end{matrix}
				\right .
			\end{align*}
			}
		\item[c.]{
			\begin{align*}
				\left \{
				\begin{matrix}
				x+y &=10 \\
				-2x-2y &=-20\\
				\end{matrix}
				\right .
			\end{align*}
			}
		\item[d.]{
			\begin{align*}
			\left \{ \begin{matrix}
			x+y+z &= 10 \\
			x-y+z &= 4 \\
			x+y-z&=2\\
			\end{matrix}
			\right .
			\end{align*}		
			}
		\item[e.]{
			\begin{align*}
			\left \{ \begin{matrix}
			2x+y+z &= 10 \\
			x-y+z &= 6 \\
			-3x-y-z&=2\\
			\end{matrix}
			\right .
			\end{align*}
			}
	\end{itemize}
\end{exercise}

\begin{exercise}
	Classifique os sistemas do Exercício \ref{ex55} em \textit{Sistema Possível e Determinado - SPD}, \textit{Sistema Possível e Indeterminado - SPI} ou \textit{Sistema Impossível - SI}.
\end{exercise}