\section{Exercícios}\index{Exercícios}

\begin{exercise}
	Faça uma representação gráfica dos conjuntos numéricos.
\end{exercise}

\begin{exercise}
	Qual a diferença entre Equação e Função?
\end{exercise}

\begin{exercise}
	Resolva a equação $2x-5=0$ considerando os conjuntos:
	\begin{itemize}
		\item[a.]{$\mathbb{N}$}
		\item[b.]{$\mathbb{Z}$}
		\item[c.]{$\mathbb{Q}$}
		\item[d.]{$\mathbb{R}$}
		\item[e.]{$\mathbb{C}$}
	\end{itemize}
\end{exercise}

\begin{exercise}
	Resolva a equação $x^2+4=0$ considerando os conjuntos:
	\begin{itemize}
		\item[a.]{$\mathbb{N}$}
		\item[b.]{$\mathbb{Z}$}
		\item[c.]{$\mathbb{Q}$}
		\item[d.]{$\mathbb{R}$}
		\item[e.]{$\mathbb{C}$}
	\end{itemize}
\end{exercise}

\begin{exercise}
	Determine os conjuntos Domínio e Imagem para cada uma das funções abaixo:
	\begin{itemize}
		\item[a.]{$3x+4=0$}
		\item[b.]{$f(x) = x^2 + 6$}
		\item[c.]{$f(x) = \frac{1}{x}$}
		\item[d.]{$f(x) = \sqrt{2x}$}
		\item[e.]{$f(x) = \sqrt{-4x}$}
		\item[f.]{$f(x) = \frac{4}{\sqrt{x^2-4}}$}
	\end{itemize}
\end{exercise}

\begin{exercise}
	Determine matematicamente se $f(x) = 3x + 6$ é crescente ou decrescente.
\end{exercise}

\begin{exercise}
	Calcule as raízes das funções abaixo e represente graficamente.
	\begin{itemize}
		\item[a.]{$f(x) = 3x - 6$}
		\item[b.]{$g(x) = x^2-5x+6$}
		\item[c.]{$f(y) = 5x - 2$}
		\item[d.]{$f(x) = x^2-4x+4$}
		\item[e.]{$g(w) = w^2+4$}
	\end{itemize}
\end{exercise}