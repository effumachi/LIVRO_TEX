\chapterimage{chapter_head_2.pdf} % Chapter heading image

\chapter{Funções}
\section{Introdução}\index{Introdução}


\indent O estudo de funções é importante entretanto, no dia a dia da maior parte das pessoas, este estudo não faz sentido, num primeiro momento. Porém não é difícil imaginar que em toda a vida de uma pessoa ela esteja direta, ou indiretamente, ligada com as funções. Desde a elaboração de uma receita de um bolo, a quilometragem que um determinado veículo pode fazer com um tanque de combustível, a \textit{Receita}, o \textit{Custo} e o \textit{Lucro} de uma empresa.
\\
\indent Considere os ingredientes necessários para se fazer um (uma receita) bolo (hipotético):
\\

\begin{itemize}
	\item{4 ovos;}
	\item{1 litro de leite;}
	\item{1 kg de farinha.}
\end{itemize}
\vspace{.5cm}

Como mencionado, os ingredientes acima estão relacionadas para se fazer \textit{uma receita} ou \textit{um bolo}. Se o indivíduo deseja fazer \textit{duas receitas}, ou \textit{dois bolos}, a quantidade dos ingredientes deve dobrar! Assim:
\\

\begin{itemize}
	\item{4 ovos $\times$ 2 $\Rightarrow$ 8 ovos;}
	\item{1 litro de leite $\times$ 2 $\Rightarrow$ 2 litros de leite;}
	\item{1 kg de farinha $\times$ 2 $\Rightarrow$ 2 kg de farinha.}
\end{itemize}
\vspace{.5cm}

Se uma empresa fabrica um produto por um custo de $\mathrm{R}\$ 10,00$ a unidade e o vende a $\mathrm{R}\$ 15,00$ a unidade e tem custos de aluguel, água, energia elétrica, entre outros que totalizam $\mathrm{R}\$ 1000,00$ por mês é possível escrever sua função \textit{Custo, Receita e Lucro} da seguinte forma:

\begin{ceqn}
	\begin{align*}
	\mathrm{R}(x) &= 15,00 \cdot x \\
	\mathrm{C}(x) &= 10,00 \cdot x +1000,00 \\
	\mathrm{L}(x) &= 5,00 \cdot x -1000,00
	\end{align*}
\end{ceqn}
\vspace{.5cm}
onde $x$ representa a quantidade de itens produzidos e vendidos.

\section{Domínio, imagem, relação e função}\index{Domínio, imagem, relação e função}

Esta seção se inicia com a ideia de \textit{Relação}. A relação deve acontecer entre dois conjuntos numéricos, $\mathbb{A}$ e $\mathbb{B}$, por exemplo.

A \textit{Relação} poderá receber o nome de \textit{Função} se, e somente se, todo elemento do conjunto de partida ($\mathbb{A}$, por exemplo), tenha um, apenas um, correspondente no conjunto de destino ($\mathbb{B}$), através de uma lei. Matematicamente, uma função pode ser escrita assim:

\begin{ceqn}
	\begin{align*}
	f: \mathbb{A} \rightarrow \mathbb{B}, \forall x \in \mathbb{A}, \, \exists y \in \mathbb{B} \,\, / \,\, y=f(x)
	\end{align*}
\end{ceqn}
\\
Vale notar que $x$ é a \textit{variável independente} e $y$ é a \textit{variável dependente} pois depende de $x$. De modo geral, o primeiro conjunto da definição de uma função está relacionado com a \textit{variável independente}, pois é o conjunto de partida e a \textit{variável dependente} está relacionada com o segundo conjunto, pois é o conjunto de chegada.

Como o objetivo deste material é trabalhar com funções, apenas, não será abordado as relações de modo geral, sendo assim deve ser definido o Domínio ($\mathrm{D}$) e Imagem ($\mathrm{Im}$) de uma função ($f$), sendo:

\begin{definition}[Domínio de uma função]
	O domínio de uma função é um conjunto numérico com os valores possíveis que podem ser atribuídos à \textit{variável independente}.
\end{definition}
\vspace{.5cm}
\begin{definition}[Imagem de uma função]
	A imagem de uma função é um conjunto numérico contendo o resultado de cada elemento do domínio aplicado na função.
\end{definition}
\vspace{.5cm}

\begin{example}
	Considere os conjuntos $\mathbb{A}=\{1,2,3\}$ e $\mathbb{B}=\{2,4,6,8\}$ e a função $f: \mathbb{A} \rightarrow \mathbb{B} \,\, / \,\, y=f(x)=2x$. Determine o Domínio e a Imagem de $f$.

\vspace{.5cm}
\textit{\textbf{Solução:}}

\vspace{.5cm}
O conjunto de partida é o conjunto $\mathbb{A}$ então, ele é o \textit{Domínio} da função ($\mathrm{D}(f)$). Os elementos do domínio são atribuídos à \textit{variável independente}, $x$.

Os elementos da \textit{Imagem} da função $f$ são obtidos da seguinte maneira:
\vspace{.5cm}
\begin{itemize}
	\item{$x=1 \Rightarrow f(1)=2\cdot 1 = 2$}
	\item{$x=2 \Rightarrow f(2)=2\cdot 1 = 4$}
	\item{$x=3 \Rightarrow f(3)=2\cdot 1 = 6$}
\end{itemize}

\vspace{.5cm}
Assim, o conjunto \textit{Imagem} é: $\mathrm{Im}=\{2,4,6\}$. Verifica-se que o conjunto imagem não é o conjunto $\mathbb{B}$ pois são os elementos que possuem um elemento $x$ do Domínio. Como o $8$ do conjunto $\mathbb{B}$ não possui um antecessor do conjunto $\mathbb{A}$, ele não faz parte da Imagem.
\end{example}

As funções, de modo geral, tem o conjunto dos \textit{Números Reais} como \textit{Domínio} e \textit{Imagem}, porém existem funções que o Domínio deve ser determinado, ou seja, será um subconjunto dos Reais.

\begin{example}
	Determine o Domínio e a Imagem da função $f(x)=\frac{1}{x-3}$

\vspace{.5cm}
\textit{\textbf{Solução:}}

\textit{Determinação do Domínio}: É possível verificar que a função possui uma divisão e a variável independente está no denominador. Este tipo de situação requer que o \textit{denominador seja diferente de zero}, pois, caso contrário, haveria uma indeterminação. Seguindo o analisado, é possível fazer:

\begin{ceqn}
	\begin{align*}
	x-3 &\neq 0 \\
	x-3 +3 & \neq 0+3 \\
	x &\neq 3
	\end{align*}
\end{ceqn}
Logo, $\mathrm{D}(f)=(-\infty,3) \wedge (3,+\infty)$ ou $\mathrm{D}(f)= \mathbb{R}-\{3\}$ ou $\mathrm{D}(f)=\{x \in \mathbb{R}/ x \neq 3\}$

\vspace{.5cm}
\textit{Determinação da Imagem}: Para a determinação da imagem é necessário analisar os extremos do domínio. Assim:

\vspace{.5cm}
\begin{itemize}
	\item{$x \rightarrow -\infty \Rightarrow f(x) \rightarrow 0$: Se $x$ aproxima-se do infinito negativo, o denominador torna-se um número muito grande e, consequentemente, a divisão aproxima-se de zero;}
	\item{$x \rightarrow -3^{-} \Rightarrow f(x) \rightarrow -\infty$: Se $x$ aproxima-se do três negativo pela esquerda, o denominador torna-se um número muito pequeno negativo e, consequentemente, a divisão aproxima-se de menos infinito;}
	\item{$x \rightarrow -3^{+} \Rightarrow f(x) \rightarrow \infty$: Se $x$ aproxima-se do três negativo pela direita, o denominador torna-se um número muito pequeno positivo e, consequentemente, a divisão aproxima-se de mais infinito;}
	\item{$x \rightarrow \infty \Rightarrow f(x) \rightarrow 0$: Se $x$ aproxima-se do infinito positivo, o denominador torna-se um número muito grande e, consequentemente, a divisão aproxima-se de zero.}
\end{itemize}

\vspace{.5cm}
Do exposto anteriormente, a Imagem nunca assumirá o valor $0$ (zero), assim a imagem será $\mathrm{Im}(f)=\mathbb{R}^{*}$ ou $\mathrm{Im}(f)=\mathbb{R}-\{0\}$
\end{example}

\vspace{.5cm}
Muitos símbolos foram usados no exemplo anterior porém, os mesmos, serão explicados posteriormente no Capítulo \ref{limites}.

\begin{example}
	Determine o Domínio e a Imagem da função $f(x)=\sqrt{x+2}$.

\vspace{.5cm}
\textit{\textbf{Solução:}}

\textit{Domínio}: Os valores que podem ser colocados na função são valores positivos (funções reais) e, nesse caso, o zero. Valores negativos dentro da raíz quadrada resultam em resultados pertencentes aos \textit{Número Complexos}, logo:

\begin{ceqn}
	\begin{align*}
	x+2 & \geq 0 \\
	x+2-2 &\geq 0-2 \\
	x &\geq -2
	\end{align*}
\end{ceqn}

Assim, $\mathrm{D}(f)=[-2,+\infty)$ ou $\mathrm{D}(f)=\{x \in \mathbb{R}/x \geq -2\}$
\end{example}

\section{Raízes de uma função}\index{Raízes de uma função}

As raízes de uma função $f(x)$ pode ser calculada ao fazer $f(x)=0$, ou seja, são os pontos em que a função "corta" o eixo da variável independente. Nos casos de polinômios, o número de raízes é igual ao maior grau da função.
Cada tipo de função possui uma metodologia diferente para o cálculo das suas raízes assim, será visto nas seções subsequentes.

\section{Funções de uma variável real}\index{Funções de uma variável real}

Difíceis são os processos que dependem apenas de uma variável, ou seja, ao estudar o consumo de combustível de um veículo não pode ser pensado somente na qualidade do combustível, mas deve ser pensado na alinhamento e pressão dos pneus, da temperatura, das condições da rodovia, das manutenções, da qualidade do óleo do motor, do modo como o veículo é conduzido (esta variável bem estudada resulta em economia de 30\% no consumo de combustível), entre outros fatores.

Como visto na seção anterior, "geralmente" se atribui $x$ a \textit{variável independente} e $y$ ou $f(x)$ a \textit{variável dependente}, assim, tem-se duas variáveis. Mas como é possível ter duas variáveis sendo que a seção trata de \textit{funções de uma variável}?

A resposta é simples: Funções de uma variável é o nome dado às funções que possuem uma, e apenas uma, \textit{variável independente}.

\begin{example}
	A área do quadrado pode ser representada por uma função de uma variável. Como a área do quadrado é a multiplicação dos dois lados, e os quatro lados são iguais, então:
	
	\begin{ceqn}
		\begin{align*}
		A_Q(l)=l \cdot l = l^2
		\end{align*}
	\end{ceqn}

onde $l$ é a medida do lado do quadrado.
\end{example}

\begin{example}
	A área de um retângulo é uma função de duas variáveis, pois sua área é a multiplicação de dois lados consecutivos. Como os lados do retângulo são iguais aos seus lados opostos, e não necessariamente precisam ter os quatro lados iguais, então:
	
	\begin{ceqn}
		\begin{align*}
		A_R(a,b)=a \cdot b
		\end{align*}
	\end{ceqn}

onde $a$ e $b$ são as medidas dos lados de um retângulo.
\end{example}

\vspace{.5cm}


\section{Funções lineares}\index{Funções lineares}


As \textit{funções lineares} ou \textit{funções do primeiro grau} ou \textit{funções de grau um} são funções do tipo:

\begin{ceqn}
	\begin{align*}
	f(x) = ax +b
	\end{align*}
\end{ceqn}

\vspace{.5cm}
Onde:
\begin{itemize}
	\item{$f$: nome da função;}
	\item{$x$: variável independente;}
	\item{$f(x)$: variável dependente;}
	\item{$a$: coeficiente angular;}
	\item{$b$: coeficiente linear.}
\end{itemize}

\vspace{.5cm}

Um ponto importante para observar é o coeficiente angular. Este valor permite identificar se a função é \textit{crescente} ($a>0$) ou \textit{decrescente} ($a<0$). Caso $a=0$, tem-se uma função constante igual a $f(x)=b$, ou seja, para funções do primeiro grau a condição é que ($a \ne 0$)

De modo geral, uma função $f(x)$ é crescente quando para quaisquer dois valores $x_1$ e $x_2$ pertencentes ao $\mathrm{Dom}(f)$, se $x_2 > x_1$, então $f(x_2)>f(x_1)$. Se $x_2>x_1 \Rightarrow f(x_2)<f(x_1)$ então a função será decrescente.


\begin{example} Determine se a função $f(x)=2x+4$ é crescente ou decrescente.

\textit{\textbf{Solução:}}

	É possível verificar que o coeficiente angular é igual a $2$ e é maior do que zero, no entanto, tomando $x_1=0$ e $x_2=2$ tem-se $f(x_1)=f(0)=2\cdot0+4=4$ e $f(x_2)=f(2)=2\cdot2+4=8$, assim, $x_2>x_1\Rightarrow f(x_2)>f(x_1)$ logo, $f(x)$ é crescente.

\end{example}

\begin{example} Determine se a função $f(x)=-x+4$ é crescente ou decrescente.
	
	\textit{\textbf{Solução:}}
	
	É possível verificar que o coeficiente angular é igual a $-1$ e é menor do que zero, no entanto, tomando $x_1=0$ e $x_2=2$ tem-se $f(x_1)=f(0)=-1\cdot0+4=4$ e $f(x_2)=f(2)=-1\cdot2+4=2$, assim, $x_2>x_1\Rightarrow f(x_2)<f(x_1)$ logo, $f(x)$ é decrescente.
	
\end{example}

\section{Raíz de funções lineares}\index{Raíz de funções lineares}

Como mencionado anteriormente, para determinar a raíz de um função é necessário fazer $f(x)=0$. Como o maior grau das funções lineares é igual a "um" então, ter-se-á, apenas, uma raíz.

\begin{example} Calcule a raíz da função $f(x)=-x+4$
	
	\textit{\textbf{Solução:}}
	
	Fazendo $f(x)=0$, tem-se:
	\begin{ceqn}
		\begin{align*}
		f(x)&=0 \\
		-x+4&=0 \\
		-x+4-4&=0-4 \\
		(-1) \cdot \,\,\,\, -x&=-4\,\,\,\, \cdot (-1) \\
		x&=4
		\end{align*}
	\end{ceqn}
	
	Assim, $x=4$ é a raíz da função $f(x)=-x+4$.
\end{example}

\section{Gráficos de funções lineares}\index{Gráficos de funções lineares}

Em construção

\section{Funções quadráticas}\index{Funções quadráticas}

As \textit{funções quadráticas} ou \textit{funções do segundo grau} ou \textit{funções de grau dois} são funções do tipo:

\begin{ceqn}
	\begin{align*}
	f(x) = ax^2 +bx+c
	\end{align*}
\end{ceqn}

\vspace{.5cm}
Onde:
\begin{itemize}
	\item{$f$: nome da função;}
	\item{$x$: variável independente;}
	\item{$f(x)$: variável dependente;}
	\item{$a$, $b$ e $c$: são coeficientes.}
\end{itemize}

\vspace{.5cm}

Um ponto importante para observar é o coeficiente $a$. Este valor permite identificar se a função possui \textit{concavidade para cima} ($a>0$) ou \textit{concavidade para baixo} ($a<0$). Caso $a=0$, tem-se uma função linear igual a $f(x)=bx+c$, ou seja, para funções do segundo grau a condição é que ($a \ne 0$).

\section{Raízes de funções quadráticas}\index{Raízes de funções quadráticas}

Para o caso de funções do segundo grau, as raízes podem ser calculadas através da, tão conhecida, fórmula de Bháskara, que é:

\begin{ceqn}
	\begin{align*}
	x = \frac{-b \pm \sqrt{b^2-4\cdot a \cdot c}}{2\cdot a}
	\end{align*}
\end{ceqn}

O termo dentro da raíz quadrada, $b^2-4ac$, é conhecido como delta $\Delta$.

\begin{example}\label{caso1}
	Calcule a raíz da função $f(x)=x^2-5x+6$
	
	\textit{\textbf{Solução:}}
	
	Fazendo $f(x)=0$, tem-se:
	\begin{ceqn}
		\begin{align*}
		f(x)&=0 \\
		x^2-5x+6&=0 \\
		\end{align*}
	\end{ceqn}
	
	O próximo passo é identificar os valores dos coeficientes $a$, $b$ e $c$. Para a equação acima, tem-se:
	
	\begin{ceqn}
		\begin{align*}
		a=1 \,\,\,\, b=-5 \,\,\,\, c=6
		\end{align*}
	\end{ceqn}

Determinado os valores dos coeficientes, calcula-se o valor de $\Delta$:
	
	\begin{ceqn}
		\begin{align*}
		\Delta &= b^2-4ac \\
		&=(-5)^2-4 \cdot 1 \cdot 6 \\
		&= (-5)\cdot (-5)-24 \\
		&= 25 - 24 \\
		\Delta &=1 \,\,\,\,\,\,\,\,\,\,\,\,\,\,\,\,\, (\mathrm{Caso 1:}\,\, \Delta>0)
		\end{align*}
	\end{ceqn}
	
	Nesta parte, deverá ser substituído os valores dos coeficientes e o $\Delta$ calculado:
	
	\begin{ceqn}
		\begin{align*}
		x &= \frac{-b\pm \sqrt{\Delta}}{2a} \\
		&= \frac{-(-5)\pm\sqrt{1}}{2\cdot 1} \\
		&= \frac{5 \pm 1}{2} \Rightarrow \\
		x_1 &= \frac{5+1}{2}=\frac{6}{2} \Rightarrow x_1 = 3 \\
		x_2 &= \frac{5-1}{2}=\frac{4}{2} \Rightarrow x_2 = 2 
		\end{align*}
	\end{ceqn}

Assim, as raízes podem ser representadas através do conjunto solução $\mathbb{S}=\{2,3\}$

\end{example}

\begin{example}\label{caso2}
	Calcule a raíz da função $f(x)=x^2-4x+4$
	
	\textit{\textbf{Solução:}}
	
	Fazendo $f(x)=0$, tem-se:
	\begin{ceqn}
		\begin{align*}
		f(x)&=0 \\
		x^2-4x+4&=0 \\
		\end{align*}
	\end{ceqn}
	
	O próximo passo é identificar os valores dos coeficientes $a$, $b$ e $c$. Para a equação acima, tem-se:
	
	\begin{ceqn}
		\begin{align*}
		a=1 \,\,\,\, b=-4 \,\,\,\, c=4
		\end{align*}
	\end{ceqn}
	
	Determinado os valores dos coeficientes, calcula-se o valor de $\Delta$:
	
	\begin{ceqn}
		\begin{align*}
		\Delta &= b^2-4ac \\
		&=(-4)^2-4 \cdot 1 \cdot 4 \\
		&= (-4)\cdot (-4)-16 \\
		&= 16-16 \\
		\Delta &=0 \,\,\,\,\,\,\,\,\,\,\,\,\,\,\,\,\, (\mathrm{Caso 2:}\,\, \Delta=0)
		\end{align*}
	\end{ceqn}
	
	Nesta parte, deverá ser substituído os valores dos coeficientes e o $\Delta$ calculado:
	
	\begin{ceqn}
		\begin{align*}
		x &= \frac{-b\pm \sqrt{\Delta}}{2a} \\
		&= \frac{-(-4)\pm\sqrt{0}}{2\cdot 1} \\
		&= \frac{4 \pm 0}{2} \Rightarrow \\
		x_1 &= \frac{4+0}{2}=\frac{4}{2} \Rightarrow x_1 = 2 \\
		x_2 &= \frac{4-0}{2}=\frac{4}{2} \Rightarrow x_2 = 2 
		\end{align*}
	\end{ceqn}
	
	Assim, as raízes podem ser representadas através do conjunto solução $\mathbb{S}=\{2\}$
	
\end{example}

\begin{example}\label{caso3} Calcule a raíz da função $f(x)=x^2-6x+13$
	
	\textit{\textbf{Solução:}}
	
	Fazendo $f(x)=0$, tem-se:
	\begin{ceqn}
		\begin{align*}
		f(x)&=0 \\
		x^2-6x+13&=0 \\
		\end{align*}
	\end{ceqn}
	
	O próximo passo é identificar os valores dos coeficientes $a$, $b$ e $c$. Para a equação acima, tem-se:
	
	\begin{ceqn}
		\begin{align*}
		a=1 \,\,\,\, b=-6 \,\,\,\, c=13
		\end{align*}		
	\end{ceqn}
	
	Determinado os valores dos coeficientes, calcula-se o valor de $\Delta$:
	
	\begin{ceqn}
		\begin{align*}
		\Delta &= b^2-4ac \\
		&=(-6)^2-4 \cdot 1 \cdot 13 \\
		&= (-6)\cdot (-6)-52 \\
		&= 36 - 52 \\
		\Delta &=-16 \,\,\,\,\,\,\,\,\,\,\,\,\,\,\,\,\, (\mathrm{Caso 3:}\,\, \Delta<0)
		\end{align*}
	\end{ceqn}
	
	Nesta parte, deverá ser substituído os valores dos coeficientes e o $\Delta$ calculado:
	
	\begin{ceqn}
	\begin{align*}
	x &= \frac{-b\pm \sqrt{\Delta}}{2a} \\
	&= \frac{-(-6)\pm \sqrt{-16}}{2\cdot 1} \xRightarrow{\text{\ref{complexo}}} \\
	&= \frac{6 \pm \sqrt{16 \cdot (-1)}}{2} \\
	&= \frac{6\pm \sqrt{16} \cdot \sqrt{-1}}{2} \\
	&= \frac{6\pm 4 \cdot i}{2} \Rightarrow \\
	x_1 &= \frac{6+4i}{2} = 3+2i \\
	x_2 &= \frac{6-4i}{2} = 3-2i \\
	\end{align*}
\end{ceqn}

	Assim, as raízes podem ser representadas através do conjunto solução $\mathbb{S}=\{3 \pm 2i\}$
	
\end{example}

\section{Gráficos de funções quadráticas}\index{Gráficos de funções quadráticas}

Os gráficos de funções de segunda grau podem apresentar 6 formas diferentes, que são:

\begin{itemize}
	\item{Caso 1: $a > 0$ e $\Delta > 0 \Rightarrow$ Parábola com concavidade para \textbf{cima} e duas \textbf{raízes reais distintas};}
	\item{Caso 2: $a > 0$ e $\Delta = 0 \Rightarrow$ Parábola com concavidade para \textbf{cima} e duas \textbf{raízes reais iguais};}
	\item{Caso 3: $a > 0$ e $\Delta < 0 \Rightarrow$ Parábola com concavidade para \textbf{cima} e duas \textbf{raízes complexas};}
	\item{Caso 4: $a < 0$ e $\Delta > 0 \Rightarrow$ Parábola com concavidade para \textbf{baixo} e duas \textbf{raízes reais distintas};}
	\item{Caso 5: $a < 0$ e $\Delta = 0 \Rightarrow$ Parábola com concavidade para \textbf{baixo} e duas \textbf{raízes reais iguais};}
	\item{Caso 6: $a < 0$ e $\Delta < 0 \Rightarrow$ Parábola com concavidade para \textbf{baixo} e duas \textbf{raízes complexas};} \\
\end{itemize}

Para fazer o gráfico é ncessário conhecer alguns pontos fundamentais da função: \\

\begin{itemize}
	\item[1.]{\textbf{Raízes}: Todas as funções do segundo grau possuem raízes, no entanto, as raízes complexas não podem ser representadas no plano real;}
	\item[2.]{\textbf{Vértice}: Todas as funções do segundo grau possuem vértices. Os vértices nessas funções representam um ponto de \textbf{extremo}, que pode ser \textbf{máximo} ($a<0$) ou \textbf{mínimo} ($a>0$);}
	\item[3.]{\textbf{Intercepto com o eixo vertical}: Este valor implica onde a função \textit{"corta o eixo y"}.} \\
\end{itemize}

\begin{example}
	(Caso 1) Determine o gráfico da função $f(x)=x^2-5x+6$
	
	\solucao
	
	Essa função é a mesma do Exemplo \ref{caso1}, ou seja, as raízes já foram determinadas ($x_1 = 2$ e $x_2 = 3$), bem como o valor do $\Delta$ ($= 1$).\\
	
	Assim, é necessário determinar o Vértice. A determinação do mesmo pode ser feita usando:
	
	\begin{ceqn}
		\begin{align*}
		x_v &= \frac{-b}{2a} \Rightarrow x_v = \frac{-(-5)}{2 \cdot 1} = \frac{5}{2} \\
		y_v &= \frac{-\Delta}{4a} \Rightarrow y_v = \frac{-1}{4 \cdot 1} = \frac{-1}{4} \\
		\end{align*}
	\end{ceqn}
	
	Portanto, as coordenadas do Vértice são $\left (\frac{5}{2},\frac{-1}{4} \right )$.\footnote{Um modo mais fácil de representar no plano é usar a representação decimal assim, as coordenadas do vértice ficará $(2,5;-0,25)$.}\\
	
	O último passo é verificar o intercepto com o eixo vertical, ou seja, determinar o valor de $f(x)$ quando $x=0$, assim:
	
	\begin{ceqn}
		\begin{align*}
		f(x) &= x^2-5x+6 \\
		x=0 \Rightarrow f(0)&=0^2-5 \cdot 0 +6 \\
		f(0) &= 6 
		\end{align*}
	\end{ceqn}
	
	Unindo as informações:\\
	\begin{itemize}
		\item[1.]{\textbf{Raízes}: $x_1=(2,0)$ e $x_2=(3,0)$}
		\item[2.]{\textbf{Vértice}:$\left( \frac{5}{2},\frac{-1}{4} \right)$}
		\item[3.]{\textbf{Intercepto}: $(0,6)$}\\
	\end{itemize}
	
	O gráfico fica:\\
	\begin{center}
		\begin{tikzpicture}[scale=1.0,line cap=round,line join=round,>=triangle 45,x=1cm,y=1cm]
		\begin{axis}[
		x=1cm,y=1cm,
		axis lines=middle,
		grid style=dashed,
		ymajorgrids=true,
		xmajorgrids=true,
		xmin=-1.1,
		xmax=6.1,
		ymin=-1.2,
		ymax=7.5,
		xlabel=$x$,
		ylabel=$y$,
		width=10cm,
		xtick={-1,...,6},
		ytick={-1,...,7},]
		\draw[line width=2pt,color=red,smooth,samples=100,domain=-.25:5.25] plot(\x,{(\x)^(2)-5*(\x)+6});
		\begin{scriptsize}
		\draw[color=red] (5.65,6.5) node {$f(x)$};
		\draw[color=black] (2.1,0.3) node {$x_1$};
		\draw[color=black] (3,-0.7) node {Vértice};
		\draw[color=black] (3.1,0.3) node {$x_2$};
		\draw[color=black] (.5,6) node {$f(0)$};
		\end{scriptsize}
		\node [fill=black, circle, scale=0.5] at (axis cs: 0,6) {};
		\node [fill=black, circle, scale=0.5] at (axis cs: 2.5,-.25) {};
		\node [fill=black, circle, scale=0.5] at (axis cs: 2,0) {};
		\node [fill=black, circle, scale=0.5] at (axis cs: 3,0) {};
		\end{axis}
		\end{tikzpicture}
	\end{center}
	
\end{example}
\begin{example}
	(Caso 2) Determine o gráfico da função $f(x)=x^2-4x+4$
	
	\solucao
	
	Essa função é a mesma do Exemplo \ref{caso2}, ou seja, as raízes já foram determinadas ($x_1 =x_2= 2$), bem como o valor do $\Delta$ ($= 0$).\\
	
	Assim, é necessário determinar o Vértice (V). A determinação do mesmo pode ser feita usando:
	
	\begin{ceqn}
		\begin{align*}
		x_v &= \frac{-b}{2a} \Rightarrow x_v = \frac{-(-4)}{2 \cdot 1} = \frac{4}{2}=2 \\
		y_v &= \frac{-\Delta}{4a} \Rightarrow y_v = \frac{-0}{4 \cdot 1} = 0 \\
		\end{align*}
	\end{ceqn}
	
	Portanto, as coordenadas de V são $\left ( 2,0 \right )$.\\
	
	O último passo é verificar o intercepto com o eixo vertical, ou seja, determinar o valor de $f(x)$ quando $x=0$, assim:
	
	\begin{ceqn}
		\begin{align*}
		f(x) &= x^2-4x+4 \\
		x=0 \Rightarrow f(0)&=0^2-4 \cdot 0 +4 \\
		f(0) &= 4\\ 
		\end{align*}
	\end{ceqn}
	
	Unindo as informações:\\
	\begin{itemize}
		\item[1.]{\textbf{Raízes}: $x_1=x_2=(2,0)$}
		\item[2.]{\textbf{Vértice}: $(2,0)$}
		\item[3.]{\textbf{Intercepto}: $(0,4)$}\\
	\end{itemize}
	Das informações anteriores, é possível verificar que as raízes ($x_1$ e $x_2$) têm a mesma coordenada que V, ou seja, $(2,0)$. O processo, nesse caso, fica um pouco complicado pois, na realidade, existem dois pontos apenas: o intercepto e o ponto $(2,0)$.\\
	
	Para resolver esse problema, deve ser escolhido dois valores para $x$, calcular suas imagens ($f(x)$) e usá-los como complemento\footnote{Sugestão: sabendo o $x_v$, neste exemplo $x_v=2$, escolhe-se dois valores para $x$ ($x_3$ e $x_4$, por exemplo), de modo que $|x_v-x_3|=|x_v-x_4|$. Essa escolha baseia-se no fato de que, pelo vértice é possível imaginar um \textit{eixo de simetria}.}. Assim, escolhendo $x_3=1$ e $x_4=3$, tem-se:
	
	\begin{ceqn}
		\begin{align*}
		x_3=1 \Rightarrow f(1) &= 1^2-4 \cdot 1 +4 = 1 \\
		x_4=3 \Rightarrow f(3) &= 3^2-4 \cdot 3 +4 = 1 \\
		\end{align*}
	\end{ceqn}
	
	É possível chamar essas coordenadas calculadas, de pontos, $A$ e $B$, sendo assim, $A=(1,1)$ e $B=(3,1)$
	
	Logo, o gráfico fica:\\
	\begin{center}
		\begin{tikzpicture}[scale=1.4,line cap=round,line join=round,>=triangle 45,x=1cm,y=1cm]
		\begin{axis}[
		x=1cm,y=1cm,
		axis lines=middle,
		grid style=dashed,
		ymajorgrids=true,
		xmajorgrids=true,
		xmin=-0.5,
		xmax=4.6000000000000005,
		ymin=-0.5000013006060611,
		ymax=4.9,
		xlabel=$x$,
		ylabel=$y$,
		width=10cm,
		xtick={-1,0,...,5},
		ytick={-1,0,...,5},]
		\draw[line width=2pt,color=red,smooth,samples=100,domain=-0.1:4.1] plot(\x,{(\x)^(2)-4*(\x)+4});
		\begin{scriptsize}
		\draw[color=red] (3.35,4.2) node {$f(x)$};
		\draw[color=black] (.5,4) node {$f(0)$};
		\end{scriptsize}
		\node [fill=black, circle, scale=0.5] at (axis cs: 0,4) {};
		\node [fill=black, circle, scale=0.5] at (axis cs: 1,1) {};
		\node [fill=black, circle, scale=0.5, pin=75:{A}] at (axis cs: 1,1) {};
		\node [fill=black, circle, scale=0.5, pin=105:{B}] at (axis cs: 3,1) {};
		\node [fill=black, circle, scale=0.1, pin=60:{$x_2$}] at (axis cs: 2,0) {};
		\node [fill=black, circle, scale=0.1, pin=120:{$x_1$}] at (axis cs: 2,0) {};
		\node [fill=black, circle, scale=0.5, pin=90:{V}] at (axis cs: 2,0) {};
		\node [fill=black, circle, scale=0.5] at (axis cs: 3,1) {};
		\end{axis}
		\end{tikzpicture}
	\end{center}
	
\end{example}

\begin{example}
	(Caso 3) Determine o gráfico da função $f(x)=x^2-6x+13$
	
	\solucao
	
	Essa função é a mesma do Exemplo \ref{caso3}, ou seja, as raízes já foram determinadas ($x_1 = 3-2i$ e $x_2 = 3+2i$), bem como o valor do $\Delta$ ($= -16$).\\
	
	A determinação do vértice pode ser feita usando:
	
	\begin{ceqn}
		\begin{align*}
		x_v &= \frac{-b}{2a} \Rightarrow x_v = \frac{-(-6)}{2 \cdot 1} = \frac{6}{2} = 3 \\
		y_v &= \frac{-\Delta}{4a} \Rightarrow y_v = \frac{-(-16)}{4 \cdot 1} = \frac{16}{4}=4 \\
		\end{align*}
	\end{ceqn}
	
	Portanto, as coordenadas do Vértice são $(3,4))$.\\
	
	O último passo é verificar o intercepto com o eixo vertical, ou seja, determinar o valor de $f(x)$ quando $x=0$, assim:
	
	\begin{ceqn}
		\begin{align*}
		f(x) &= x^2-6x+13 \\
		x=0 \Rightarrow f(0)&=0^2-6 \cdot 0 +13 \\
		f(0) &= 13 
		\end{align*}
	\end{ceqn}
	
	Unindo as informações:\\
	\begin{itemize}
		\item[1.]{\textbf{Raízes}: $x_1=3-2i$ e $x_2=3+2i$}
		\item[2.]{\textbf{Vértice}: $(3,4)$}
		\item[3.]{\textbf{Intercepto}: $(0,13)$}\\
	\end{itemize}
	Das informações anteriores, é possível verificar que as raízes ($x_1$ e $x_2$) pertencem aos conjunto dos números complexos e não tem representação no plano real. O processo, nesse caso, fica um pouco complicado pois, na realidade, existem dois pontos apenas: o intercepto e o vértice $(3,4)$.\\
	
	Para resolver esse problema, deve ser escolhido dois valores para $x$, calcular suas imagens ($f(x)$) e usá-los como complemento. Assim, escolhendo $x_3=2$ e $x_4=4$, tem-se:
	
	\begin{ceqn}
		\begin{align*}
		x_3=2 \Rightarrow f(2) &= 2^2-6 \cdot 2 +13 = 5 \\
		x_4=4 \Rightarrow f(4) &= 4^2-6 \cdot 4 +13 = 5 \\
		\end{align*}
	\end{ceqn}
	
	É possível chamar essas coordenadas calculadas, de pontos, $A$ e $B$, sendo assim, $A=(2,5)$ e $B=(4,5)$
	
	O gráfico fica:\\
	\begin{center}
		\begin{tikzpicture}[line cap=round,line join=round,>=triangle 45,x=1.5cm,y=.9cm]
		\begin{axis}[
		x=1.5cm,y=.9cm,
		axis lines=middle,
		grid style=dashed,
		ymajorgrids=true,
		xmajorgrids=true,
		xmin=-1.2,
		xmax=6.9,
		ymin=-1.1,
		ymax=15.,
		xlabel=$x$,
		ylabel=$y$,
		width=10cm,
		xtick={-1,0,...,7},
		ytick={-1,0,...,14},]
		\draw[line width=2pt,color=red,smooth,samples=100,domain=-.25:6.5] plot(\x,{(\x)^(2)-6*(\x)+13});
		\node [fill=black, circle, scale=0.5, pin=45:{$f(0)$}] at (axis cs: 0,13) {};
		\node [fill=black, circle, scale=0.5, pin=75:{A}] at (axis cs: 2,5) {};
		\node [fill=black, circle, scale=0.5, pin=105:{B}] at (axis cs: 4,5) {};
		\node [fill=black, circle, scale=0.5, pin=-45:{V}] at (axis cs: 3,4) {};
		\end{axis}
		\end{tikzpicture}
	\end{center}
	
\end{example}

Os casos em que o valor de $a$ forem menores que zero, o processo é semelhante, no entanto, o resultado será uma parábola com a \textbf{concavidade para baixo}.

\section{Exercícios}\index{Exercícios}

\begin{exercise}
	Qual a diferença entre Equação e Função?
\end{exercise}

\begin{exercise}
	Resolva a equação $x^2+4=0$ considerando os conjuntos:
	\begin{itemize}
		\item[a.]{$\mathbb{N}$}
		\item[b.]{$\mathbb{Z}$}
		\item[c.]{$\mathbb{Q}$}
		\item[d.]{$\mathbb{R}$}
		\item[e.]{$\mathbb{C}$}
	\end{itemize}
\end{exercise}

\begin{exercise}
	Calcule as raízes das funções abaixo e represente graficamente.
	\begin{itemize}
		\item[a.]{$f(x) = 3x - 6$}
		\item[b.]{$g(x) = x^2-5x+6$}
		\item[c.]{$f(y) = 5x - 2$}
		\item[d.]{$f(x) = x^2-4x+4$}
		\item[e.]{$g(w) = w^2+4$}
	\end{itemize}
\end{exercise}