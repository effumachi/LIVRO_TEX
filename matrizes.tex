%----------------------------------------------------------------------------------------
%	CHAPTER 1
%----------------------------------------------------------------------------------------
\chapterimage{chapter_head_2.pdf} % Chapter heading image

\chapter{Matrizes}

\section{Conceitos iniciais}\index{Conceitos iniciais}


Uma matriz pode ser definida como uma \textit{coleção de elementos} e cada elemento possui uma posição definida através dos indicadores \textit{i} e \textit{j}, sendo \textit{i} o indicador para a linha e \textit{j} o indicador para a coluna. Os valores possíveis para \textit{i, j} são valores naturais\footnote{O autor considera números naturais os positivos diferentes de zero pois, \textit{naturalmente}, não dizemos \textit{"Tenho ZERO Ferrari"} para expressar a ideia de que \textit{"NÃO TENHO uma Ferrari"}.} ($i, j \in \mathbb{N} = \{1,2,3,... \}$), ou inteiros positivos diferente de zero ($i, j \in \mathbb{Z}_{+}^{*} = \{ 1, 2, 3, ... \}$).
A representação de um elemento é dada por:

\begin{ceqn}
	\begin{align*}
	a_{ij}
	\end{align*}
\end{ceqn}

no caso acima, o elemento possui o nome \textit{a} com uma posição qualquer \textit{i , j}. Os nomes dos elementos costumam seguir o
mesmo nome da matriz que os contém, mas em letra minúscula. Assim, se a matriz tiver o nome $\mathbb{A}$ seus elementos serão $a_{i j}$,
se a matriz se chamar $\mathbb{B}$ seus elementos serão $b_{i j}$ e assim sucessivamente. Os nomes das matrizes são em letras maiúsculas
com uma barra dupla em sua construção. 
Existem, também, os indicadores para as matrizes; esses indicadores são chamados de \textit{Ordem da matriz}, ou seja, representa o número
de linhas e de colunas. Logo, uma matriz chamada A com 2 linhas e 3 colunas é escrita da seguinte forma:

\begin{ceqn}
	\begin{align*}
	\mathbb{A}_{2\mathrm{{x}}3}=\left[\begin{array}{ccc}
	a_{11} & a_{12} & a_{13}\\
	a_{21} & a_{22} & a_{23}
	\end{array}\right]
	\end{align*}
\end{ceqn}

Desse modo temos a representação de uma \textit{matriz} e de todos os seus \textit{elementos}. É possível verificar que o número de
elementos da matriz é 6. De modo geral, o número de elementos ($\#_{elementos}$) de uma matriz será o resultado da \textit{multiplicação} do número de \textit{linhas} pelo número de \textit{colunas}, assim:

\begin{ceqn}
	\begin{align*}
	\#_{elementos}=i\cdot j
	\end{align*}
\end{ceqn}

\begin{example}
Determine o número de elementos da matriz $\mathbb{A}_{3\mathrm{x}5} $.\\
Na matriz do enunciado tem-se $i=3$ (linhas) e $j=5$ (colunas).
Logo, o número de elementos dessa matriz será $i \cdot j = 3 \cdot 5 = 15\, \mathrm{elementos}$
\end{example}


\begin{example}
	Considere a matriz
	
	\begin{align*}
	\mathbb{B}_{3\mathrm{x}2} & =\begin{bmatrix}3 & 7\\
	1 & 6\\
	10 & 4
	\end{bmatrix}
	\end{align*}
	
	escreva todos os elementos colocando sua posição corretamente.

	Como o nome da matriz dada é $\mathbb{B}$, os elementos serão escritos pela letra minúscula do nome da matriz, ou seja, os elementos serão
	$b_{i j}$.
\begin{itemize}
	\item O primeiro elemento está na \textit{linha} 1 e \textit{coluna} 1,
	então: $b_{1 1} = 3$
	\item O segundo elemento está na \textit{linha} 1 e \textit{coluna} 2,
	então: $b_{1 2} = 7$
	\item O terceiro elemento está na \textit{linha} 2 e \textit{coluna} 1,
	então: $b_{2 1} = 1$
	\item O quarto elemento está na \textit{linha} 2 e \textit{coluna} 2,
	então: $b_{2 2} = 6$
	\item O quinto elemento está na \textit{linha} 3 e \textit{coluna} 1,
	então: $b_{3 1} = 10$
	\item O sexto elemento está na \textit{linha} 3 e \textit{coluna} 2, então:
	$b_{3 2} = 4$
\end{itemize}
\end{example}
As representações das matrizes podem ser da seguinte forma, ainda:

\begin{ceqn}
	\begin{align*}
	\mathbb{M}_{2}\quad\mathrm{ou}\quad\mathbb{M}_{3}\quad\mathrm{ou\quad\mathbb{M}_{4}\,...}
	\end{align*}
\end{ceqn}

Nesse caso, as matrizes são chamadas de \textit{matrizes quadradas} pois elas possuem o número de linhas igual ao número de colunas, ou
seja, $i=j$. O primeiro caso é uma matriz quadrada de ordem 2, a segunda é uma matriz quadrada de ordem 3, a terceira é uma matriz
quadrada de ordem 4, e assim sucessivamente.
\begin{example}
	Matriz de segunda ordem, ou matriz quadrada de ordem 2:

\begin{ceqn}
	\begin{align*}
		\mathbb{B}_{2}=\begin{bmatrix}b_{11} & b_{12}\\
	b_{21} & b_{22}
	\end{bmatrix}
	\end{align*}
\end{ceqn}

\end{example}
%
\begin{example}
	Matriz quadrada de ordem 3:
	
\begin{ceqn}
	\begin{align*}
		\mathbb{B}_{3}=\begin{bmatrix}b_{11} & b_{12} & b_{13}\\
	b_{21} & b_{22} & b_{23}\\
	b_{31} & b_{32} & b_{33}
	\end{bmatrix}
	\end{align*}
\end{ceqn}

\end{example}


\section{Igualdade entre matrizes}

Duas, ou mais, matrizes são iguais quando, nas respectivas posições,
todos os elementos são iguais.
\begin{example}
	Considere as matrizes
	
\begin{ceqn}
	\begin{align*}
		\mathbb{A}_{2\mathrm{x}3}=\begin{bmatrix}1 & 2 & 3\\
	6 & 5 & 4
	\end{bmatrix}\,\,\,\,\,\,\,\mathrm{e\,\,\,\,\,\,\,\mathbb{B}_{2\mathrm{x}3}=\begin{bmatrix}1 & 2 & 3\\
		6 & 5 & 4
		\end{bmatrix}}
	\end{align*}
\end{ceqn}
	
	As matrizes $\mathbb{A}$ e $\mathbb{B}$ são iguais, pois $a_{11}=b_{11}=1$
	e $a_{12}=b_{12}=2$ e $a_{13}=b_{13}=3$ e $a_{21}=b_{21}=6$ e $a_{22}=b_{22}=5$
	e $a_{23}=b_{23}=4$
\end{example}
%
\begin{example}
	Considere as matrizes
	
\begin{ceqn}
	\begin{align*}
		\mathbb{A}_{2\mathrm{x}3}=\begin{bmatrix}1 & 2 & 3\\
	6 & 5 & 4
	\end{bmatrix}\,\,\,\,\,\,\,\mathrm{e\,\,\,\,\,\,\mathbb{B}_{2\mathrm{x}3}=\begin{bmatrix}1 & 2 & 3\\
		6 & 5 & 2
		\end{bmatrix}}
	\end{align*}
\end{ceqn}
	
	As matrizes $\mathbb{A}$ e $\mathbb{B}$ NÃO são iguais. Embora $a_{11}=b_{11}=1$
	e $a_{12}=b_{12}=2$ e $a_{13}=b_{13}=3$ e $a_{21}=b_{21}=6$ e $a_{22}=b_{22}=5$,
	tem-se que $a_{23}\ne b_{23}$
\end{example}
%
\begin{example}
	Determine os valores de $x$ e $y$ de modo que as matrizes abaixo
	sejam iguais:

\begin{ceqn}
	\begin{align*}
	\mathbb{A}_{2}=\begin{bmatrix}1 & x+10\\
	3 & 5
	\end{bmatrix}\quad\mathrm{e}\quad\mathbb{B}_{2}=\begin{bmatrix}1 & 25\\
	y-7 & 5
	\end{bmatrix}
	\end{align*}
\end{ceqn}
\\
	Analisando as matrizes, verifica-se que $a_{11}=b_{11}=1$ e $a_{22}=b_{22}=5$.
	O elemento $a_{12}=x+10$ e $b_{12}=25$ devem ser iguais, ou seja:

\begin{ceqn}
	\begin{align*}
	a_{12} &= b_{12} \\
	x+10 &= 25 \\
	x+10-10 &= 25-10 \\
	x &= 15 \\
	\end{align*}
\end{ceqn}

De modo análogo, o elemento $a_{21}=3$ e $b_{21}=y-7$ devem ser iguais, assim:

\begin{ceqn}
	\begin{align*}
		a_{21} &=b_{21} \\
		3 &=y-7 \\
		3+7 &=y-7+7 \\
		10 &=y \\
	\end{align*}
\end{ceqn}

	Substituindo os valores calculados, $x$ e $y$, tem-se:

\begin{ceqn}
	\begin{align*}
		\mathbb{A}_{2} &= \mathbb{B}_{2}\\
		\begin{bmatrix}1 & 15+10\\
		3 & 5
		\end{bmatrix} &= \begin{bmatrix}1 & 25\\
		10-7 & 5
		\end{bmatrix}\\
		\begin{bmatrix}1 & 25\\
		3 & 5
		\end{bmatrix} &= \begin{bmatrix}1 & 25\\
		3 & 5
		\end{bmatrix}
	\end{align*}
\end{ceqn}

\end{example}

\section{Adição e subtração de matrizes}

A adição, ou subtração, de matrizes só podem ser feitas se a(s) matriz(es) possuir/possuírem a mesma ordem. Assim, $\mathbb{A}_{i\mathrm{x}j}$
e $\mathbb{B}_{i\mathrm{x}j}$ podem ser somadas e/ou subtraídas da seguinte maneira:


\begin{ceqn}
	\begin{align*}
		\mathbb{A}_{i\mathrm{x}j}\pm\mathbb{B}_{i\mathrm{x}j} & = \begin{bmatrix}a_{11} & a_{12} & \cdots & a_{1j}\\
	a_{21} & a_{22} & \cdots & a_{2j}\\
	\vdots & \vdots & \ddots & \vdots\\
	a_{i1} & a_{i2} & \cdots & a_{ij}
	\end{bmatrix}\pm\begin{bmatrix}b_{11} & b_{12} & \cdots & b_{1j}\\
	b_{21} & b_{22} & \cdots & b_{2j}\\
	\vdots & \vdots & \ddots & \vdots\\
	b_{i1} & b_{i2} & \cdots & b_{ij}
	\end{bmatrix}\\
	& = \begin{bmatrix}a_{11}\pm b_{11} & a_{12}\pm b_{12} & \cdots & a_{1j}\pm b_{1j}\\
	a_{21}\pm b_{21} & a_{22}\pm b_{22} & \cdots & a_{2j}\pm b_{2j}\\
	\vdots & \vdots & \ddots & \vdots\\
	a_{i1}\pm b_{i1} & a_{i2}\pm b_{i2} & \cdots & a_{ij}\pm b_{ij}
	\end{bmatrix}
	\end{align*}
\end{ceqn}

\begin{example}
	\video \, Considere as matrizes abaixo e efetue a adição entre elas, ou seja, $\mathbb{A}+\mathbb{B}$
	
	\begin{ceqn}
		\begin{align*}
			\mathbb{A}_{2\mathrm{x}3}=\begin{bmatrix}1 & 3 & -5\\
		2 & 6 & 4
		\end{bmatrix}\qquad\mathrm{e}\qquad\mathbb{B}_{2\mathrm{x}3}=\begin{bmatrix}3 & 5 & 6\\
		-2 & 4 & 10
		\end{bmatrix}
		\end{align*}
	\end{ceqn}
\\

	Partindo das matrizes dadas, podemos escrever:


\begin{ceqn}
	\begin{align*}
		\mathbb{A}_{2\mathrm{x}3}+\mathbb{B}_{2\mathrm{x}3} & =\\
	\begin{bmatrix}1 & 3 & -5\\
	2 & 6 & 4
	\end{bmatrix}+\begin{bmatrix}3 & 5 & 6\\
	-2 & 4 & 10
	\end{bmatrix} & =  \begin{bmatrix}1+3 & 3+5 & -5+6\\
	2+(-2) & 6+4 & 4+10
	\end{bmatrix}\\
	& =  \begin{bmatrix}4 & 8 & 1\\
	0 & 10 & 14
	\end{bmatrix}
	\end{align*}
\end{ceqn}


\doutor \url{https://www.youtube.com/watch?v=VeP6FNgG9bg}
\end{example}

\section{Multiplicação de um escalar por uma matriz}

Esse tipo de multiplicação pode ser entendido como \textit{um número multiplicando uma matriz}, sendo assim, matematicamente, basta multiplicar todos os elementos da matriz pelo número que está multiplicando a matriz.

Nesse momento a notação se faz necessária, pois numa expressão do tipo $x \cdot y$ não fica evidente quem é a matriz e quem é o escalar,
ou se são duas matrizes, ou se são dois escalares! Então dado um escalar $a$ e uma matriz $\mathbb{A}_{ij}$ a multiplicação pode ser escrita
como:


\begin{ceqn}
	\begin{align*}
		a\cdot\mathbb{A}_{i\mathrm{x}j} & =  a\cdot\begin{bmatrix}a_{11} & a_{12} & \cdots & a_{1j}\\
	a_{21} & a_{22} & \cdots & a_{2j}\\
	\vdots & \vdots & \ddots & \vdots\\
	a_{i1} & a_{i2} & \cdots & a_{ij}
	\end{bmatrix}\\
	& =  \begin{bmatrix}a\cdot a_{11} & a\cdot a_{12} & \cdots & a\cdot a_{1j}\\
	a\cdot a_{21} & a\cdot a_{22} & \cdots & a\cdot a_{2j}\\
	\vdots & \vdots & \ddots & \vdots\\
	a\cdot a_{i1} & a\cdot a_{i2} & \cdots & a\cdot a_{ij}
	\end{bmatrix}
	\end{align*}
\end{ceqn}

Embora seja um pouco confuso, devido a quantidade de letras ``a'', isso foi escolhido propositalmente para o entendimento de um conceito: o escalar $a$ é diferente de todos os elementos da matriz pois não possui indicadores, ou seja, subíndice.


\begin{example}
	Calcule o valor de $2\cdot \mathbb{A}+4\cdot \mathbb{B}$ sendo
	
	\begin{ceqn}
		\begin{align*}
			\mathbb{A}_{2\mathrm{x}3}=\begin{bmatrix}1 & 3 & -5\\
		2 & 6 & 4
		\end{bmatrix}\qquad\mathrm{e}\qquad\mathbb{B}_{2\mathrm{x}3}=\begin{bmatrix}3 & 5 & 6\\
		-2 & 4 & 10
		\end{bmatrix}
		\end{align*}
	\end{ceqn}

	Para calcular, basta substituir os valores das matrizes, multiplicar pelo escalar e somar as matrizes. Mesmo sabendo o procedimento para calcular a expressão dada, é importante verificar a ordem das matrizes envolvidas na expressão, pois após a multiplicação pelo escalar, haverá uma soma e, como visto anteriormente, só é possível somar matrizes se elas forem da mesma ordem!
	
	\begin{ceqn}
		\begin{align*}
		2\cdot\mathbb{A}+4\cdot\mathbb{B} & =\\
		2\cdot\begin{bmatrix}1 & 3 & -5\\
		2 & 6 & 4
		\end{bmatrix}+4\cdot\begin{bmatrix}3 & 5 & 6\\
		-2 & 4 & 10
		\end{bmatrix} & =  \begin{bmatrix}2\cdot1 & 2\cdot3 & 2\cdot(-5)\\
		2\cdot2 & 2\cdot6 & 2\cdot4
		\end{bmatrix}+\begin{bmatrix}4\cdot3 & 4\cdot5 & 4\cdot6\\
		4\cdot(-2) & 4\cdot4 & 4\cdot10
		\end{bmatrix}\\
		& =  \begin{bmatrix}2 & 6 & -10\\
		4 & 12 & 8
		\end{bmatrix}+\begin{bmatrix}12 & 20 & 24\\
		-8 & 16 & 40
		\end{bmatrix}\\
		& =  \begin{bmatrix}2+12 & 6+20 & -10+24\\
		4+(-8) & 12+16 & 8+40
		\end{bmatrix}\\
		& =  \begin{bmatrix}14 & 26 & 14\\
		-4 & 28 & 48
		\end{bmatrix}
		\end{align*}
	\end{ceqn}

\end{example}


\section{Multiplicação entre matrizes}

Nesse momento é necessário fazer uma análise mais criteriosa sobre as matrizes. A multiplicação de matrizes NÃO é comutativa, ou seja,
$\mathbb{A} \cdot \mathbb{B} \ne \mathbb{B} \cdot \mathbb{A}$. Outro ponto importante na multiplicação entre matrizes é que dada duas matrizes,
o número de colunas da primeira deve ser igual ao número de linhas da segunda matriz. Assim, sejam as matrizes $\mathbb{A}_{i\mathrm{x}j}$
e $\mathbb{B}_{n\mathrm{x}m}$, a multiplicação $\mathbb{A} \cdot \mathbb{B}$ existirá se, se somente se, $j=n$ e a multiplicação $\mathbb{B} \cdot \mathbb{A}$ existirá se, e somente se, $m=i$.
\begin{example}
	\video \, Calcule $\mathbb{A} \cdot \mathbb{B}$ para as matrizes abaixo:
	
	\begin{ceqn}
		\begin{align*}
			\mathbb{A}_{1\mathrm{x}3}=\begin{bmatrix}1 & 2 & 3\end{bmatrix}\quad\mathrm{e}\quad\mathbb{B}_{3\mathrm{x}2}=\begin{bmatrix}3 & 4\\
		2 & 6\\
		5 & 8
		\end{bmatrix}
		\end{align*}
	\end{ceqn}

	A primeira matriz é a matriz $\mathbb{A}$ e a segunda matriz é a $\mathbb{B}$. Analisando o número de linhas da primeira matriz e
	o número de colunas da segunda matriz, verifica-se que são iguais a 3, logo, existe a multiplicação proposta no enunciado. Calculando-a:
	
	
	\begin{ceqn}
		\begin{align*}
		\mathbb{A}_{1\mathrm{x}3}\cdot\mathbb{B}_{3\mathrm{x}2} & =\\
		\begin{bmatrix}1 & 2 & 3\end{bmatrix}\cdot\begin{bmatrix}3 & 4\\
		2 & 6\\
		5 & 8
		\end{bmatrix} & =  \begin{bmatrix}1\cdot3+2\cdot2+3\cdot5 & 1\cdot4+2\cdot6+3\cdot8\end{bmatrix}\\
		& =  \begin{bmatrix}3+4+15 & 4+12+24\end{bmatrix}\\
		& =  \begin{bmatrix}22 & 40\end{bmatrix}_{1\mathrm{x}2}
		\end{align*}
	\end{ceqn}
	
	é possível verificar que a matriz resultante possui a ordem $1\mathrm{x}2$ que é, exatamente, o número de linhas da primeira matriz e o número de colunas da segunda matriz.
	
\doutor \url{https://www.youtube.com/watch?v=VeP6FNgG9bg}
\end{example}


\begin{example}
	\video \, Calcule a multiplicação $\mathbb{M} \cdot \mathbb{N}$ sendo
	
	\begin{ceqn}
		\begin{align*}
			\mathbb{M}_{2}=\begin{bmatrix}-3 & 6\\
		2 & 7
		\end{bmatrix}\quad\mathrm{e}\quad\mathbb{N}_{2}=\begin{bmatrix}2 & -1\\
		4 & 3
		\end{bmatrix}
		\end{align*}
	\end{ceqn}
	
	Como as matrizes são quadradas de ordem 2, é possível a multiplicação proposta no enunciado, logo
	
	\begin{ceqn}
		\begin{align*}
		\mathbb{M}\cdot\mathbb{N} & =  \begin{bmatrix}-3 & 6\\
		2 & 7
		\end{bmatrix}\cdot\begin{bmatrix}2 & -1\\
		4 & 3
		\end{bmatrix}\\
		& = \begin{bmatrix}(-3)\cdot2+6\cdot4 & (-3)\cdot(-1)+6\cdot3\\
		2\cdot2+7\cdot4 & 2\cdot(-1)+7\cdot3
		\end{bmatrix}\\
		& =  \begin{bmatrix}-6+24 & 3+18\\
		4+28 & -2+21
		\end{bmatrix}\\
		& =  \begin{bmatrix}18 & 21\\
		32 & 19
		\end{bmatrix}
		\end{align*}
	\end{ceqn}
	
	é possível verificar que a ordem resultante da multiplicação entre matrizes quadradas é, também, uma matriz quadrada com a mesma ordem.
	Outro ponto a se notar é que foi omitido o índice de ordem das matrizes na resolução pois, como são matrizes quadradas, não se faz necessário.
	
\doutor \url{https://www.youtube.com/watch?v=VeP6FNgG9bg}
\end{example}


\section{Tipos de matrizes}
\begin{itemize}
	\item \textit{\textbf{Matriz Linha}}: é a matriz que possui somente uma linha. É conhecida, também, como \textit{vetor linha} ou apenas \textit{\textbf{vetor}}; 
	\item \textit{\textbf{Matriz Coluna}}: é a matriz que possui apenas uma coluna e também é conhecida como \textit{vetor coluna};
	\item \textit{\textbf{Matriz Quadrada}}: como visto anteriormente, a matriz quadrada é aquela em que o número de linhas é igual ao número de colunas;
	\begin{itemize}
		\item \textit{Diagonal Principal}: a diagonal principal de uma matriz quadrada é aquela em que os elementos possuem os indicadores iguais, ou seja, $i=j \Rightarrow a_{ii}$;
		\item \textit{Diagonal Secundária}: é a diagonal formada pelos elementos $a_{ij}$ de modo que $i+j=\textrm{ordem}+1$.
		\item \textit{Triangular Superior}: matriz onde os elementos \textbf{acima} da diagonal principal e os da diagonal principal são diferentes de
		zero, isto é, $a_{ij}=0 \, , \, i>j$;
		\begin{ceqn}
			\begin{align*}
					\begin{bmatrix}1 & 2 & 3\\
			0 & 5 & 1\\
			0 & 0 & 2
			\end{bmatrix}
			\end{align*}
		\end{ceqn}
		\item \textit{Triangular Inferior}: matriz onde os elementos \textbf{abaixo} da diagonal principal e os da diagonal principal são diferentes de
		zero, isto é, $a_{ij}=0 \, , \, i<j$
		\begin{ceqn}
			\begin{align*}
			\begin{bmatrix}1 & 0 & 0\\
			2 & 5 & 0\\
			4 & 6 & 2
			\end{bmatrix}
			\end{align*}
		\end{ceqn}
		\item \textit{Diagonal}: matriz onde os elementos da diagonal principal são diferentes de zero, isto é, $a_{ij} \ne 0 \, ,\, i=j$.
		\begin{ceqn}
			\begin{align*}
			\begin{bmatrix}1 & 0 & 0\\
			0 & 5 & 0\\
			0 & 0 & 2
			\end{bmatrix}
			\end{align*}
		\end{ceqn}
		\item \textit{Identidade}: matriz onde todos os elementos da diagonal principal são iguais a 1, ou seja, $a_{ij}=1\, ,\, i=j$. Essa matriz recebe uma representação peculiar, $\mathbb{I}_n$ ou apenas $\mathbb{I}$
		\begin{ceqn}
			\begin{align*}
			\begin{bmatrix}1 & 0 & 0\\
			0 & 1 & 0\\
			0 & 0 & 1
			\end{bmatrix}
			\end{align*}
		\end{ceqn}
	\end{itemize}
	\item \textit{\textbf{Matriz Oposta}}: dada uma matriz $\mathbb{A}$ e uma matriz $\mathbb{B}$, é possível dizer que a matriz $\mathbb{B}$
	é oposta a matriz $\mathbb{A}$ se todos os elementos de $\mathbb{B}$ forem os elementos simétricos de $\mathbb{A}$
	\begin{ceqn}
		\begin{align*}
		\mathbb{A}=\begin{bmatrix}1 & 10 & -7\\
		2 & -1 & -8\\
		3 & 5 & 9
		\end{bmatrix}\quad\Rightarrow\quad\mathbb{B}=\begin{bmatrix}-1 & -10 & 7\\
		-2 & 1 & 8\\
		-3 & -5 & -9
		\end{bmatrix}
		\end{align*}
	\end{ceqn}
	\item \textit{\textbf{Matriz Transposta}}: dada uma matriz $\mathbb{A}$, a sua transposta, ou seja, $\mathbb{A}^t$ terá as suas colunas formadas
	pelas linhas de $\mathbb{A}$. Vale ressaltar que $({\mathbb{A}^t})^t = \mathbb{A}$
	\begin{ceqn}
		\begin{align*}
		\mathbb{A}=\begin{bmatrix}1 & 2 & 3\\
		9 & 8 & 7
		\end{bmatrix}\Rightarrow\mathbb{A}^{t}=\begin{bmatrix}1 & 9\\
		2 & 8\\
		3 & 7
		\end{bmatrix}
		\end{align*}
	\end{ceqn}
	\item \textit{\textbf{Matriz Simétrica}}: dada uma matriz $\mathbb{A}$, é possível dizer que $\mathbb{A}$ é simétrica se $\mathbb{A}=\mathbb{A}^t$.
	\begin{ceqn}
		\begin{align*}
		\mathbb{A}=\begin{bmatrix}1 & 2 & 3\\
		2 & -1 & 5\\
		3 & 5 & 9
		\end{bmatrix}\quad\Rightarrow\quad\mathbb{A}^{t}=\begin{bmatrix}1 & 2 & 3\\
		2 & -1 & 5\\
		3 & 5 & 9
		\end{bmatrix}=\mathbb{A}
		\end{align*}
	\end{ceqn}
	\item \textit{\textbf{Matriz Antissimétrica}}: dada uma matriz $\mathbb{A}$, é possível dizer que $\mathbb{A}$ é antissimétrica se $-\mathbb{A}=\mathbb{A}^t$. Aqui, se faz necessário mais uma observação: necessariamente $a_{ij}=0, \,i=j$.
	\begin{ceqn}
		\begin{align*}
		\mathbb{A}=\begin{bmatrix}0 & 2 & -3\\
		-2 & 0 & -5\\
		3 & 5 & 0
		\end{bmatrix}\quad\Rightarrow\quad\mathbb{A}^{t}=\begin{bmatrix}0 & -2 & 3\\
		2 & 0 & 5\\
		-3 & -5 & 0
		\end{bmatrix}=-\mathbb{A}
		\end{align*}
	\end{ceqn}
	\item \textit{\textbf{Matriz Nula}}: é uma matriz onde todos os elementos são iguais a $0$ (zero).
	\begin{ceqn}
		\begin{align*}
		\mathbf{0}_{2\mathrm{x}4}=\begin{bmatrix}0 & 0 & 0 & 0\\
		0 & 0 & 0 & 0
		\end{bmatrix},\quad\mathbf{0}_{2}=\begin{bmatrix}0 & 0\\
		0 & 0
		\end{bmatrix},\quad\mathbf{0}_{3}=\begin{bmatrix}0 & 0 & 0\\
		0 & 0 & 0\\
		0 & 0 & 0
		\end{bmatrix}
		\end{align*}
	\end{ceqn}
	\item \textit{\textbf{Matriz Ortogonal}}: dada uma matriz $\mathbb{A}$, \textit{inversível}, ela será ortogonal se obedecer $\mathbb{A}^t = \mathbb{A}^{-1}$.\footnote{Será visto mais adiante como calcular a inversa de uma matriz, no entanto, a ideia é partir de uma equação matricial. Dadas as matrizes $\mathbb{A}, \mathbb{X}$ e a matriz identidade $\mathbb{I}$. A equação base é $\mathbb{A}\cdot \mathbb{X} = \mathbb{I}$ e o objetivo é determinar a matriz $\mathbb{X}$. O leitor habituado com os métodos tradicionais poderia imaginar que para resolver essa equação bastaria \textit{passar a matriz $\mathbb{A}$ dividindo}, no entanto, não existe divisão de matrizes, muito menos \textit{passa para lá}. O correto é determinar uma matriz de modo que a multiplicação dela por sua inversa, resulte no elemento neutro da multiplicação entre matrizes; o elemento neutro da multiplicação entre matrizes é a matriz identidade!}
	\begin{ceqn}
		\begin{align*}
		\mathbb{A}=\begin{bmatrix}\frac{1}{2} & -\frac{\sqrt{3}}{2}\\
		\frac{\sqrt{3}}{2} & \frac{1}{2}
		\end{bmatrix}\quad\mathbb{A}^{t}=\begin{bmatrix}\frac{1}{2} & \frac{\sqrt{3}}{2}\\
		-\frac{\sqrt{3}}{2} & \frac{1}{2}
		\end{bmatrix}\Rightarrow\mathbb{A}\cdot\mathbb{A}^{t}=\begin{bmatrix}1 & 0\\
		0 & 1
		\end{bmatrix}
		\end{align*}
	\end{ceqn}
\end{itemize}

