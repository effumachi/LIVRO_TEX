%----------------------------------------------------------------------------------------
%	SYGNATURE PAGE
%----------------------------------------------------------------------------------------
\newpage
\thispagestyle{plain}
%\addcontentsline{toc}{chapter}{\protect\numberline{}Sobre o autor}
\phantomsection
\addcontentsline{toc}{chapter}{\protect\numberline{}Prefácio}

\newcommand*{\titulointro}{\addvspace{12pt}\Huge\fontfamily{yes}\selectfont}
\newcommand*{\descricaointro}{\fontfamily{yes}\selectfont}
{\titulointro\textbf{Prefácio}}
\\
\\
{\descricaointro{\indent Este material foi elaborado exclusivamente pelo autor e tem como objetivo atender alunos dos mais variados cursos, Administração, Ciências Contábeis, Tecnólogos, Matemática, Engenharia entre outros. Este material está \textit{\textbf{continuamente em desenvolvimento}} e é distribuído em partes, ou seja, apenas os capítulos necessários aos estudantes de uma determinada disciplina. Portanto, se quiser a versão mais nova e completa, contate o autor enviando um e-mail para \textit{effumachi@gmail.com}. %\cite{flemming2007,hoffmann2008,tan2008,jacques2010,bonafini2011,murolo2011,oliveira2016,barbosa2017,castanheira2017,panonceli2017,souza2018,munaretto2018,boldrini1986,steinbruch2009,franco2016,fernandes2017,domingues2018}.
\vspace{.5cm}

Os exemplos que possuem uma versão em vídeo possuirão o símbolo \video \, e um \textit{link} que encaminhará ao canal do "Doutor Exatas", no Youtube (\url{https://www.youtube.com/channel/UCqGy3MbhdZsWGGBBg7yUGRw}). O leitor pode, ainda, encontrar o conteúdo no Facebook
(\url{https://www.facebook.com/doutorexatas/}).
\vspace{.5cm}

%As ementas das disciplinas de matemática para os cursos de Administração e Ciências Contábeis devem ter elementos fundamentais para desenvolver competências e habilidades dos discentes, deste modo, este livro abordará os seguintes conteúdos:
%\vspace{.5cm}
%\begin{itemize}
%	\item{Revisão sobre conjuntos numéricos e equações de grau 1;}
%	\item{Funções de uma variável real: Domínio, Imagem, funções lineares, quadráticas e seus gráficos;}
%	\item{Limites de funções de uma variável real;}
%	\item{Derivadas de funções, regras e aplicações;}
%	\item{Integrais indefinidas e definidas, cálculo de áreas e aplicações.}
%\end{itemize}
%\vspace{.5cm}

A abordagem adotada neste material é fornecer um pouco de teoria (porém suficiente), e em seguida com a resolução de exemplos.

\vspace{1cm}
Bons estudos.}}